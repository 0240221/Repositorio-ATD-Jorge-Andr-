\documentclass[12pt]{book}
\usepackage[utf8]{inputenc}
\usepackage[spanish]{babel}
\usepackage{amsmath, amssymb, amsfonts}
\usepackage{amsthm}

%%%%%%%%%%%%%%%%%%%%%%%%%%%%%%%%%%%%
% Esto lo quitaremos después
%\usepackage[paperwidth=275.9mm, paperheight=279.4mm]{geometry}
\usepackage{todonotes}
%%%%%%%%%%%%%%%%%%%%%%%%%%%%%%%%%%%%%%%%

\author{Larruz Castillo Oscar André\\
Jorge Eduardo Gutiérrez Jiménez}
\title{Notas Análisis Topológico de Datos}
\date{\today}

\newtheorem{theorem}{Teorema}[chapter]
\newtheorem*{theorem*}{Teorema}
\newtheorem{lemma}[theorem]{Lema}
\newtheorem{corollary}[theorem]{Corolario}
\newtheorem{proposition}[theorem]{Proposición}
\newtheorem{example}[theorem]{Ejemplo}
\newtheorem{definition}[theorem]{Definición}
\newtheorem{reminder}[theorem]{Recordatorio}
\newtheorem{notation}[theorem]{Notación}

\begin{document}
	\frontmatter
	\maketitle
    
    	\begin{reminder}
        		Un grupo es una tercia ordenada $(A,*,0_A)$, donde se cumple que:
	\begin{enumerate}
		\item [1.]
		$A$ es un conjunto 
		\item [1.]
		$0_A$ es un elemento de $A$
		\item [3.]
		* es una función cuyo dominio es $A \times A$, y cuyo contradominio es $A$, y que cumple que:
		\begin{enumerate}
			\item [a)]
			* es asociativa.
			\item [b)]
			Para toda $a\in A$ se cumple que $*(0_A,a)=*(a,0_A)=a$.
			\item [c)]
			Para toda $a\in A$, existe $b\in A$ tal que $*(b,a)=*(a,b)=0_A$.
		\end{enumerate}
	\end{enumerate}
    	\end{reminder}

	\begin{reminder}
		Un grupo $(A,*;0_A)$ es llamado abeliano si la función $*$ es conmutativa.
	\end{reminder}
	
	\begin{reminder}
		Sea $(A,*,0_A)$ un grupo. Decimos que $B$ es subrgupo de $A$ si ocurre que $B \subseteq A$, $B \neq \varnothing$, y 
		$(B,*_{\restriction A},0_A)$ es un grupo; y lo denotamos como $B \leq A$.
	\end{reminder}

	\begin{notation}
		A partir de ahora, a los grupos los denotaremos solo con su primer elemento, al elemento $0_A$ como $0$, a la función $*$ con la suma, y 
		denotaremos $\sum_{i=1}^{n} a = a^n$.
	\end{notation}

	\begin{reminder}
		Sea $A$ un grupo, y sea $a \in A$. Definimos el orden de $a$ como el mínimo natural tal que $a^n = 0$, en caso de que dicho natural no exista,
		decimos que el orden de $a$ es infinito; denotamos el orden de $a$ como $ord(a)$
	\end{reminder}

    	\begin{definition}
        		La torsión de un grupo abeliano $A$ es el subgrupo $B$ de $A$ que cumple que, para todo
        		$b \in B$, tenemos que $ord(b) <\infty$.
        		Denotamos así a este grupo como $T(A)$.
    	\end{definition}
    
    	\begin{definition}
        		Un grupo abeliano $A$ es llamado libre de torsión si $T(A) = \{0\}$, o equivalentemente, si para todo $a \in A$, $ord (a)= \infty$
    	\end{definition}

\end{document}