\documentclass[12pt]{book}
\usepackage[utf8]{inputenc}
\usepackage[spanish]{babel}
\usepackage{amsmath, amssymb, amsfonts}
\usepackage{amsthm}

%%%%%%%%%%%%%%%%%%%%%%%%%%%%%%%%%%%%
% Esto lo quitaremos después
%\usepackage[paperwidth=275.9mm, paperheight=279.4mm]{geometry}
\usepackage{todonotes}
%%%%%%%%%%%%%%%%%%%%%%%%%%%%%%%%%%%%%%%%

\author{Larruz Castillo Oscar André\\
Jorge Eduardo Gutiérrez Jiménez}
\title{Notas Análisis Topológico de Datos}
\date{\today}

\newtheorem{theorem}{Teorema}[chapter]
\newtheorem*{theorem*}{Teorema}
\newtheorem{lemma}[theorem]{Lema}
\newtheorem{corollary}[theorem]{Corolario}
\newtheorem{proposition}[theorem]{Proposición}
\newtheorem{example}[theorem]{Ejemplo}
\newtheorem{definition}[theorem]{Definición}

\begin{document}
    \frontmatter
    \maketitle
    \todo{Mejoren la redacción y agregeuen el párrafo que les pido.}
    
    \begin{definition}
        \todo{Agreguen algunos párrafos con las definiciones de grupo,
            grupo abeliano y orden. No tiene por qué ser definiciones como esta sino como recordatorio}
        La torsión de un grupo abeliano $A$ es el subgrupo de A $B \leq A$ que cumple que para todo
        $b \in B, ord(b)=n, n \in \mathbb{N}$,
        \todo{Dentro del grupo matemático anterior hay comas no matemática. ¿Es para todo $b$ y $n$?}
        y se denota por $T(A)$
    \end{definition}
    
    \begin{definition}
        Un grupo abeliano A es llamado libre de torsión si $T(A) = \{0\}$.
        Equivalentemente\todo{Esta oración no tiene sentido. el punto sobra.},
        si para todo $a \in A, ord (a)= \infty$\todo{Dentro del grupo matemático anterior hay una coma no matemática.}
    \end{definition}

\end{document}