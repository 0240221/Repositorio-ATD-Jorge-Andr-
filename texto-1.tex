\documentclass[12pt]{book}
\usepackage[utf8]{inputenc}
\usepackage[spanish]{babel}
\usepackage{amsmath, amssymb, amsfonts}
\usepackage{amsthm}


\author{Larruz Castillo Oscar André\\
	Jorge Eduardo Gutiérrez Jiménez}
\title{Notas Análisis Topológico de Datos}
\date{\today}

\newtheorem{theorem}{Teorema}[chapter]
\newtheorem*{theorem*}{Teorema}
\newtheorem{lemma}[theorem]{Lema}
\newtheorem{corollary}[theorem]{Corolario}
\newtheorem{proposition}[theorem]{Proposición}
\newtheorem{example}[theorem]{Ejemplo}
\newtheorem{definition}[theorem]{Definición}

\begin{document}
\frontmatter
\maketitle

\begin{definition}
	La torsión de un grupo abeliano $A$ es el subgrupo de A $B \leq A$ que cumple que para todo $b \in B, ord(b)=n, n \in \mathbb{N}$, y se denota por
	$T(A)$
\end{definition}

\begin{definition}
	Un grupo abeliano A es llamado libre de torsión si\\ $T(A) = \{0\}$. Equivalentemente, si para todo $a \in A, ord (a)= \infty$
\end{definition}

\end{document}